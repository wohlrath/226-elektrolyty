\section*{Teoretická část}
Podle stupně disociace dělíme elektrolyty na silné (stupeň disociace blízký jedné) a slabé (velmi nízký stupeň disociace).

Měříme závislost měrné voivosti $\sigma$ roztoku na molární koncentraci $c_M$.
Definujeme molární konduktivitu \cite{skripta}
\begin{equation}
\Lambda = \frac{\sigma}{c_M} \,.
\end{equation}

Molární konduktivita se mění s koncentrací, neboť je na ní závislá pohyblivost iontů.
Pro silné a slabé elektrolyty je koncentrační závislost různá.
U silných elektrolytů lze závislost popsat empirickým vztahem \cite{skripta}
\begin{equation} \label{e:silne}
\Lambda = \Lambda_0 - k \cdot \sqrt{c_M} \,,
\end{equation}
kde $k$ je konstanta a $\Lambda_0$ je limitní molární konduktivita při nekonečném zředění.